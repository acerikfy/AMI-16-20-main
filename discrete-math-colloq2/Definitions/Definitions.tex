\documentclass[a4paper,12pt]{article}

\usepackage{header}

\begin{document}
	\title{Дискретная математика. Коллоквиум весна 2017.\\ Определения}
	\author{Ваномас}
	\maketitle
	
	%Определения даются просто как \item.
	\begin{enumerate}
	    \setcounter{enumi}{3}
	    \item
	    %4
	    События $A$ и $B$ называются независимыми, если вероятность композиции событий $p(AB)$ равна произведению вероятностей $p(A)\cdot p(B)$
	    \medskip\\
	    Свойства независимых событий: \begin{enumerate}
	        \item Если $p(B)\ne0$, то условная вероятности $p(A\,|\,B)$ равна вероятности события $p(A)$.
	        \item Если события $A$ и $B$ независимы, то события $\overline{A}$ и $B$, $A$ и $\overline{B}$ и $\overline{A}$ и $\overline{B}$ также независимы.
	    \end{enumerate}
	    \setcounter{enumi}{17}
		\item
		%18
		Свойства вычислимой функции:
		\begin{enumerate}
		    \item Если функция $f$ вычислима, то её область определения $D(f)$ является перечислимым множеством.
		    \item Если функция $f$ вычислима, то её область значений $E(f)$ является перечислимым множеством.
		    \item Если функция $f$ вычислима, то для любого перечислимого множества $X$ его образ $f(X)$ является перечислимым множеством.
		    \item Если функция $f$ вычислима, то для любого перечислимого множества $X$ его прообраз $f^{-1}(X)$ является перечислимым множеством.
		\end{enumerate}
		\item
		%19
		Множество называется \textit{разрешимым}, если для него существует разрешающий алгоритм, который на любом входе останавливается за конечное число шагов ({\itshape разрешающий алгоритм для множества --- алгоритм, получающий на вход натуральное число и определяющий, принадлежит ли оно данному множеству}).
		\item
		%20
		Множество называется \textit{перечислимым}, если все его элементы могут быть получены с помощью некоторого алгоритма.
		\item
		%21
		Свойства перечислимых множеств:
		\begin{enumerate}
		    \item Если множества $A$ и $B$ перечислимы, то их объединение $A \cup B$ и пересечение $A \cap B$ также перечислимы (\textit{отсюда следует, что объединение или пересечение конечного числа перечислимых множеств перечислимо}).
		    \item Если множество $A$ перечислимо, то оно является областью значений некоторой вычислимой функции (\textit{это также является достаточным условием перечислимости}).
		    \item Если множество $A$ перечислимо, то оно является областью определения некоторой вычислимой функции (\textit{это также является достаточным условием перечислимости}).
		\end{enumerate}
		\item
		%22
		Функция $U:\N\times\N\to\N$ называется универсальной, если для любой функции $f: \N\to\N$ существует такое $p$, что $U(p, x)=f(x)$ для любых $x$ (\textit{равенство здесь понимается в том смысле, что при любом $x$ обе функции либо принимают одинаковое значение, либо не определены}).
	\end{enumerate}
		
	
\end{document}
